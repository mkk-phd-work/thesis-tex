\section{Discussion}\label{sec:discussion}

%    \begin{itemize}
%        \item Provide a concise summary of the most important results from your analysis.
%        \item Relate the findings back to your research questions or hypotheses.
%        \item Highlight any unexpected or novel results.
%    \end{itemize}
%   \item \textbf{RQ1:} Can existing feature importance techniques be applied to multi-series and hierarchical models?


%\textcolor{red}{ TODO: Add discussion on the results and the implications for the research.}
%\textcolor{blue}{
%
%    \subsection*{Summary of Key Findings}
%    \begin{itemize}
%        \item Provide a concise summary of the most important results from your analysis.
%        \item Relate the findings back to your research questions or hypotheses.
%        \item Highlight any unexpected or novel results.
%    \end{itemize}
%    \subsection*{2. Interpretation of Results}
%    \begin{itemize}
%        \item Explain what the results mean in the context of your research objectives.
%        \item Relate your findings to the existing literature, showing how your work supports, extends, or challenges previous research.
%        \item Discuss the practical or theoretical significance of your findings. How do they contribute to the field?
%    \end{itemize}
%    \subsection*{3. Comparison with Previous Work}
%    \begin{itemize}
%        \item Compare your findings with those from other studies. Are your results consistent or different? Why might that be?
%        \item Discuss how your work fits into the broader literature and theoretical framework.
%    \end{itemize}
%    \subsection*{4. Implications of the Findings}
%    \begin{itemize}
%        \item Explain the practical or theoretical implications of your results.
%        \item Consider the broader impact of your findings on the field or on real-world applications.
%        \item If relevant, discuss policy or decision-making implications.
%    \end{itemize}
%    \subsection*{5. Limitations of the Study}
%    \begin{itemize}
%        \item Acknowledge any limitations in your research, such as constraints related to the methodology, data, or scope of the study.
%        \item Discuss how these limitations might have affected your results or interpretations.
%        \item Suggest ways these limitations could be addressed in future research.
%    \end{itemize}
%    \subsection*{6. Future Research Directions}
%    \begin{itemize}
%        \item Propose areas for future research based on your findings and the limitations you've identified.
%        \item Discuss how future studies could build on your work, including new methods, expanded datasets, or different research contexts.
%    \end{itemize}
%    \subsection*{7. Practical Applications (if applicable)}
%    \begin{itemize}
%        \item If your research has practical applications, discuss how your findings can be applied in industry, policy, or other fields.
%        \item Describe specific use cases or real-world scenarios where your work could be impactful.
%    \end{itemize}
%}

%\subsection{Expected contributions}
%
%This research is expected to contribute to the following areas:
%\begin{itemize}
%    \item Evaluation of method: Assessment of feature importance methods in hierarchical forecasting models.
%    \item Guidelines, best practices and limitations: Recommendations for explaining hierarchical forecasting models.
%    \item Tool development: Development of tools for better understanding of hierarchical models.
%\end{itemize}
%\subsection{Challenges and next steps} \label{subsec:challenges_next_steps}
%We already faced some challenges during the implementation of initial approach:
%\begin{itemize}
%    \item Proper tooling: SHAP library doesn't support models with categorical variables.
%    This is a limitation for the current library.
%    \item Visualizing the results: The SHAP library provides plotting functions, but for large number of series the plots were not working properly.
%\end{itemize}


%    \begin{itemize}
%        \item Acknowledge any limitations in your research, such as constraints related to the methodology, data, or scope of the study.
%        \item Discuss how these limitations might have affected your results or interpretations.
%        \item Suggest ways these limitations could be addressed in future research.
%    \end{itemize}


In this work, we managed to calculate the Shapley values for the predictions of a hierarchical forecasting model
with some limitations, while we also aggregated these values to different levels of the hierarchy.
By this we addressed the first two research questions.
We used a sample of a real-world dataset to evaluate the proposed method working towards the third research question.
We visualised the SHAP values at different levels of the hierarchy and provided some interpretation of the results.
To respond to the fourth research question, we plan to expand the literature review to include a wider range of XAI techniques.


%methodological limitations
This research is expected to contribute in several key areas.
First, it will provide an evaluation of feature importance methods in the context of hierarchical forecasting models.
This will help to identify what methods are most effective and how they can be applied to improve model interpretability.
Second, it aims to provide guidelines, best practices, and limitations of effectively explaining these models.
Finally, the research will support the development of tools that improve the understanding of hierarchical forecasting models
and their underlying rules and reasoning.


The limitations of our study include the handling of categorical variables in the SHAP library.
The effect of ordinal encoding that induces an order on the categorical variables may not be appropriate for all cases.
Low feature importance for the categorical variables may be due to the encoding method.
Recent research \cite{kristin_blesch_conditional_2023} proposes a method to handle categorical variables for conditional feature importance.
% data limitations
With regard to data limitations, the dataset used is simplified in multiple dimensions.
First, with aggregation of sales data at the weekly level, multiple exogenous variables such as special events could not be included.
Second, the dataset is limited to a single product category, which may not be representative of all hierarchical forecast scenarios.
Lastly, the input data was limited to the sales lag of the product without considering other products in the same category.
The independence of products in the same category may not be a realistic assumption.

During the implementation of our initial approach, we encountered several challenges.
One of the main issues was the lack of appropriate tools.
For example, the SHAP library does not support categorical features in the current version.
In addition,we faced difficulties in visualising the results;
although the SHAP library offers integrated graphing functions, these have not been effectively used to deal with
a large number of cohors, leading to errors and incomplete plots.
Although these issues are not straightforward to solve, they are a sign of unexplored areas in the field of XAI and hierarchical forecasting.

There are also potential risks that could impact research in addition to challenges.
One of the main risks is the availability of data, especially real-world datasets that include exogenous variables or demand drivers.
Synthetic datasets can be used as an alternative, but they may not capture the complexity of real-world scenarios.
The evaluation of methods is another potential risk, as it may be difficult to assess the performance of the explanation methods.
In case of application grounded evaluation, it may be difficult to find experts in the field who can provide meaningful feedback
given that each product category may require different domain knowledge\cite{doshi}.

\subsection{Future work} \label{subsec:future_work}

To address the challenges and limitations of the current research, several next steps are proposed for each part of the research.
\begin{itemize}
    \item First, the literature review will be extended to include a broader range of XAI techniques.
    Given the current context, we focus on feature importance-based evaluation, partial dependence plots, feature interaction, and other XAI techniques that should be included in the review.
    \item Data collection will be expanded to include synthetic datasets and additional real-world datasets.
    In addition, including more data from the actual dataset and forecasting on the day level can be a future direction.
    \item Evaluation and implementation of the tool for other methods will be needed.
    Conditional permutation importance can be also evaluated after implementing the method.
    \item The model implementation could be extended to include additional ML models for hierarchical forecasting.
    An additional enhancement to that would be dependent multiseries forecasting, as usually product sales are not independent
of each other, especially in the same product category.
    \item Rule extraction based on feature importance and interaction can be a future direction.
    \item After covering the methodological aspects, an empirical study with evaluation in terms of accuracy and computational efficiency is planned.
\end{itemize}


%\textbf{Next steps}
%\begin{itemize}
%%    \item Literature Review: Extend it with focus on the following aspects:
%%        \begin{itemize}
%%            \item XAI(Explainable AI) techniques and their application in demand forecasting.
%%            \item Application of other XAI methods in hierarchical and multi-series models.
%%            \item Feature importance and model reasoning in hierarchical models.
%%            \item Challenges and limitations of existing XAI methods.
%%        \end{itemize}
%%    \item Data collection: Create synthetic dataset and collect additional real-world datasets for hierarchical demand forecasting.
%%    \item Tool evaluation or implementation: Evaluate existing libraries for hierarchical forecasting and XAI techniques or implement custom tools specialized for hierarchical forecasting.
%%    \item Model implementation: Train and evaluate additional ML models for hierarchical forecasting.
%    \item Feature importance analysis: Apply XAI techniques to determine feature importance and assess their impact on forecasts.
%    \item Model reasoning: Analyze feature contributions to uncover underlying rules, especially at different hierarchy levels, and how to extract rules.
%    Synthesize the results to provide a better understanding of the model's reasoning
%
%\end{itemize}

%For future work we plan to conduct an empirical study on the applicability of feature importance methods to hierarchical forecasting models.
%For this purpose we plan to use both synthetic and real-world datasets to evaluate the methods.
%Additionally, the inclusion fo these techniques in decision support systems can be a future direction.

%\textcolor{red}{
%
%    \subsection{Conclusion}
%    \begin{itemize}
%        \item Summary of objectives and findings.
%        \item Importance of XAI in improving hierarchical forecasting models and understanding their reasoning.
%    \end{itemize}
%}

\subsection{Conclusion}\label{subsec:conclusion}
Nowadays every organisation thrives in the direction of becoming data driven.
In this context, data-driven decision making is crucial for optimising business processes to remain competitive.
This effort is supported by the use of data mining, machine learning, and AI techniques.
To avoid blindly trusting ML models,it is crucial to understand the reasoning behind their decisions.
Our goal is to demystify hierarchical forecasting models by applying XAI techniques.

This study explores the usage of SHAP values to explain the importance of features in hierarchical forecasting models.
Our preliminary results focused on the practical aspects of aggregating SHAP values at different levels of
hierarchy. This approach provides insights into the model's reasoning.
We plan to extend this work by evaluating other XAI techniques to enhance the explainability of hierarchical forecasting models.


\textbf{Acknowledgement}

This work was done in collaboration with my Ph.D. supervisor, Laura Diosan, from Babes-Bolyai University.
I am grateful for her continued support and encouragement throughout this research.




