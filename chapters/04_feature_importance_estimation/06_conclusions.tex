%Directly address how the results contribute to the field and research questions.
There are several pitfalls while evaluating feature importance of tree-based models in the context of multi-series forecasting.
Our goal was to point out some of these pitfalls and find ways to mitigate them.
We have shown that the choice of feature importance method can have a significant impact on the results.
In addition, when calculating the FI values, scaling of features can also have an impact on the results.

%Highlight novel aspects with implications
The novelty of our work lies in multiple aspects.
First, by developing and using a simulation framework for benchmarking feature importance methods, we aim to provide reproducible results and comparability of different methods.
In addition, we identify common issues in the evaluation of the importance of features in demand forecasting models.
These findings are relevant not only for the field of multi-series demand forecasting but also for other times series forecasting tasks.

%Implications for future work
Future work could involve more complex data generation processes, such as correlated time series, or the study of other data structures, such as hierarchical time series for demand forecasting.
Inverse scaling of feature contributions of SHAP given the scaling coefficients of the features could also be a potential practical research direction.

%If possible, end with a strong statement encapsulating the essence and significance of the research.
Our work contributes towards the development of trustworthy AI-driven decision support systems in demand forecasting, by providing insights into the post-hoc evaluation of tree-based models in multi-series forecasting.
