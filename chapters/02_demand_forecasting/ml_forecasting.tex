
% Machine learning for forecasting


\subsection{Demand Forecasting with machine learning} \label{subsec:demand-forecasting-with-machine-learning}
Demand forecasting is a prediction problem that aims to estimate future needs based on historical data.
Statistical forecasting methods such as ARIMA\cite{jamal_fattah_forecasting_2018,ingle2021demand} and exponential smoothing \cite{ingle2021demand} have been widely used in demand forecasting.
However, they have limitations in intermittent multi-series and hierarchical forecasting, where machine learning models have shown better performance\cite{spiliotis2022comparison}.
An important aspect also is that there may be multiple exogenous variables so-called demand drivers\cite{vandeput2023demand} that can influence the demand.
Internal factors such as price, promotions, and external factors like weather, holidays, and economic indicators can be considered as demand drivers.
These can be used as features in machine learning models to improve forecast accuracy.

Machine learning models such as tree ensembles and neural networks have been successfully applied to demand forecasting tasks\cite{spiliotis2022comparison}.
Ensemble models in general can be homogeneous with individual models of the same type or heterogeneous with models of different types.
We considered only homogeneous ensemble tree models because of the applicability of some model-specific explanation methods.
To build tree ensembles, bagging methods such as random forest\cite{leo_breiman_random_2001} can be used, which trains multiple decision trees on different subsets of the data, and the final prediction is the average of the predictions of the individual models.
In addition, boosting methods such as Gradient Boosting Machines (GBM) \cite{jerome_h_friedman_greedy_2001}, XGBoost \cite{tianqi_chen_xgboost_2016}, and LightGBM \cite{guolin_ke_highly_2017}, which train models sequentially on the residuals of the previous model, in this case using the sum of individual predictions.
In a notable forecasting competition \cite{makridakis_m5_2022}, a LightGBM model was the winner and secured four of the top five positions.


