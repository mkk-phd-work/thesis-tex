%\chapter{Demand Forecasting}
%\label{ch:demand_forecasting}


\section{Statistical Models}
\label{sec:statistical_models}


AR, ARIMA, SARIMAX models are widely used when the series is stationary and data is univariate.




\section{Machine Learning Models}
\label{sec:machine_learning_models}

% Machine learning for forecasting



\subsection{Forecasting techniques} \label{subsec:forecasting-techniques}
\begin{itemize}
    \item Single-step forecasting
    \item Multi-step forecasting
\end{itemize}

Forecasting models
\begin{itemize}
    \item Univariate
    \item Independent multi-series
    \item Dependent multi-series or multi-variate
\end{itemize}

Forecasting techniques can be divided into single-series or multi-series forecasting from the perspective of the model's input.
Single-series forecasting refers to the prediction of a single time series, while multiseries forecasting involves the prediction of multiple time series, with the same global model\cite{joachim2023demand}.
These series can be related to each other, such as sales of different products, or they can be independent, such as sales in different regions; therefore, it is important to consider the hierarchical structure of the data.



%Hierarchical forecasting

Hierarchical forecasting refers to the prediction of multiple time series that are related to each other in a hierarchical structure\cite{hyndman2018forecasting}.
It can be tackled with different single-level approaches, such as bottom-up, top-down, or middle-out\cite{hyndman2018forecasting}.
The top-down approach would involve a single series model for the total demand and then disaggregating it to the lower levels.
The middle-out and bottom-up approach would involve a multiseries model.
Grouped time-series forecasting is a special case of hierarchical forecasting, where the series are aggregated based on attributes such as product type, region, or sales channel.

\cite{vandeput2023demand} suggests three major hierarchies in demand forecasting: product hierarchy, geographical hierarchy, and time hierarchy.
The product hierarchy refers to the categorisation of products according to their attributes, such as product type, brand, or category.
The geographic hierarchy involves the division of sales regions based on geographic attributes, such as country, state, or city down to the point of sale.
Time hierarchy refers to the temporal structure of the data, such as year, month, week, day, and hour.



\section{Multi-step Forecasting}
\label{sec:multi_step_forecasting}
%\input{chapters/02_demand_forecasting/multi_step_forecasting}
\begin{itemize}
    \item Direct forecasting
    \item Recursive forecasting
    \item Multiple-output forecasting
\end{itemize}






