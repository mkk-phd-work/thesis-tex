%\chapter{Demand Forecasting}
%\label{ch:demand_forecasting}


\section{Statistical Models}
\label{sec:statistical_models}


AR, ARIMA, SARIMAX models are widely used when the series is stationary and data is univariate.




\section{Machine Learning Models}
\label{sec:machine_learning_models}

% Machine learning for forecasting





\subsection{Demand Forecasting with machine learning} \label{subsec:demand-forecasting-with-machine-learning}
Demand forecasting is a prediction problem that aims to estimate future needs based on historical data.
Statistical forecasting methods such as ARIMA\cite{jamal_fattah_forecasting_2018,ingle2021demand} and exponential smoothing \cite{ingle2021demand} have been widely used in demand forecasting.
However, they have limitations in intermittent multi-series and hierarchical forecasting, where machine learning models have shown better performance\cite{spiliotis2022comparison}.
An important aspect also is that there may be multiple exogenous variables so-called demand drivers\cite{vandeput2023demand} that can influence the demand.
Internal factors such as price, promotions, and external factors like weather, holidays, and economic indicators can be considered as demand drivers.
These can be used as features in machine learning models to improve forecast accuracy.



Machine learning models such as tree ensembles and neural networks have been successfully applied to demand forecasting tasks\cite{spiliotis2022comparison}.
Ensemble models in general can be homogeneous with individual models of the same type or heterogeneous with models of different types.
We considered only homogeneous ensemble tree models because of the applicability of some model-specific explanation methods.
To build tree ensembles, bagging methods such as random forest\cite{leo_breiman_random_2001} can be used, which trains multiple decision trees on different subsets of the data, and the final prediction is the average of the predictions of the individual models.
In addition, boosting methods such as Gradient Boosting Machines (GBM) \cite{jerome_h_friedman_greedy_2001}, XGBoost \cite{tianqi_chen_xgboost_2016}, and LightGBM \cite{guolin_ke_highly_2017}, which train models sequentially on the residuals of the previous model, in this case using the sum of individual predictions.
In a notable forecasting competition \cite{makridakis_m5_2022}, a LightGBM model was the winner and secured four of the top five positions.




\subsection{Forecasting techniques} \label{subsec:forecasting-techniques}
\begin{itemize}
    \item Single-step forecasting
    \item Multi-step forecasting
\end{itemize}

Forecasting models
\begin{itemize}
    \item Univariate
    \item Independent multi-series
    \item Dependent multi-series or multi-variate
\end{itemize}

Forecasting techniques can be divided into single-series or multi-series forecasting from the perspective of the model's input.
Single-series forecasting refers to the prediction of a single time series, while multiseries forecasting involves the prediction of multiple time series, with the same global model\cite{joachim2023demand}.
These series can be related to each other, such as sales of different products, or they can be independent, such as sales in different regions; therefore, it is important to consider the hierarchical structure of the data.



%Hierarchical forecasting

Hierarchical forecasting refers to the prediction of multiple time series that are related to each other in a hierarchical structure\cite{hyndman2018forecasting}.
It can be tackled with different single-level approaches, such as bottom-up, top-down, or middle-out\cite{hyndman2018forecasting}.
The top-down approach would involve a single series model for the total demand and then disaggregating it to the lower levels.
The middle-out and bottom-up approach would involve a multiseries model.
Grouped time-series forecasting is a special case of hierarchical forecasting, where the series are aggregated based on attributes such as product type, region, or sales channel.

\cite{vandeput2023demand} suggests three major hierarchies in demand forecasting: product hierarchy, geographical hierarchy, and time hierarchy.
The product hierarchy refers to the categorisation of products according to their attributes, such as product type, brand, or category.
The geographic hierarchy involves the division of sales regions based on geographic attributes, such as country, state, or city down to the point of sale.
Time hierarchy refers to the temporal structure of the data, such as year, month, week, day, and hour.



\section{Multi-step Forecasting}
\label{sec:multi_step_forecasting}
%\input{chapters/02_demand_forecasting/multi_step_forecasting}
\begin{itemize}
    \item Direct forecasting
    \item Recursive forecasting
    \item Multiple-output forecasting
\end{itemize}






